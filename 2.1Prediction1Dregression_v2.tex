
% Default to the notebook output style

    


% Inherit from the specified cell style.




    
\documentclass[11pt]{article}

    
    
    \usepackage[T1]{fontenc}
    % Nicer default font (+ math font) than Computer Modern for most use cases
    \usepackage{mathpazo}

    % Basic figure setup, for now with no caption control since it's done
    % automatically by Pandoc (which extracts ![](path) syntax from Markdown).
    \usepackage{graphicx}
    % We will generate all images so they have a width \maxwidth. This means
    % that they will get their normal width if they fit onto the page, but
    % are scaled down if they would overflow the margins.
    \makeatletter
    \def\maxwidth{\ifdim\Gin@nat@width>\linewidth\linewidth
    \else\Gin@nat@width\fi}
    \makeatother
    \let\Oldincludegraphics\includegraphics
    % Set max figure width to be 80% of text width, for now hardcoded.
    \renewcommand{\includegraphics}[1]{\Oldincludegraphics[width=.8\maxwidth]{#1}}
    % Ensure that by default, figures have no caption (until we provide a
    % proper Figure object with a Caption API and a way to capture that
    % in the conversion process - todo).
    \usepackage{caption}
    \DeclareCaptionLabelFormat{nolabel}{}
    \captionsetup{labelformat=nolabel}

    \usepackage{adjustbox} % Used to constrain images to a maximum size 
    \usepackage{xcolor} % Allow colors to be defined
    \usepackage{enumerate} % Needed for markdown enumerations to work
    \usepackage{geometry} % Used to adjust the document margins
    \usepackage{amsmath} % Equations
    \usepackage{amssymb} % Equations
    \usepackage{textcomp} % defines textquotesingle
    % Hack from http://tex.stackexchange.com/a/47451/13684:
    \AtBeginDocument{%
        \def\PYZsq{\textquotesingle}% Upright quotes in Pygmentized code
    }
    \usepackage{upquote} % Upright quotes for verbatim code
    \usepackage{eurosym} % defines \euro
    \usepackage[mathletters]{ucs} % Extended unicode (utf-8) support
    \usepackage[utf8x]{inputenc} % Allow utf-8 characters in the tex document
    \usepackage{fancyvrb} % verbatim replacement that allows latex
    \usepackage{grffile} % extends the file name processing of package graphics 
                         % to support a larger range 
    % The hyperref package gives us a pdf with properly built
    % internal navigation ('pdf bookmarks' for the table of contents,
    % internal cross-reference links, web links for URLs, etc.)
    \usepackage{hyperref}
    \usepackage{longtable} % longtable support required by pandoc >1.10
    \usepackage{booktabs}  % table support for pandoc > 1.12.2
    \usepackage[inline]{enumitem} % IRkernel/repr support (it uses the enumerate* environment)
    \usepackage[normalem]{ulem} % ulem is needed to support strikethroughs (\sout)
                                % normalem makes italics be italics, not underlines
    \usepackage{mathrsfs}
    

    
    
    % Colors for the hyperref package
    \definecolor{urlcolor}{rgb}{0,.145,.698}
    \definecolor{linkcolor}{rgb}{.71,0.21,0.01}
    \definecolor{citecolor}{rgb}{.12,.54,.11}

    % ANSI colors
    \definecolor{ansi-black}{HTML}{3E424D}
    \definecolor{ansi-black-intense}{HTML}{282C36}
    \definecolor{ansi-red}{HTML}{E75C58}
    \definecolor{ansi-red-intense}{HTML}{B22B31}
    \definecolor{ansi-green}{HTML}{00A250}
    \definecolor{ansi-green-intense}{HTML}{007427}
    \definecolor{ansi-yellow}{HTML}{DDB62B}
    \definecolor{ansi-yellow-intense}{HTML}{B27D12}
    \definecolor{ansi-blue}{HTML}{208FFB}
    \definecolor{ansi-blue-intense}{HTML}{0065CA}
    \definecolor{ansi-magenta}{HTML}{D160C4}
    \definecolor{ansi-magenta-intense}{HTML}{A03196}
    \definecolor{ansi-cyan}{HTML}{60C6C8}
    \definecolor{ansi-cyan-intense}{HTML}{258F8F}
    \definecolor{ansi-white}{HTML}{C5C1B4}
    \definecolor{ansi-white-intense}{HTML}{A1A6B2}
    \definecolor{ansi-default-inverse-fg}{HTML}{FFFFFF}
    \definecolor{ansi-default-inverse-bg}{HTML}{000000}

    % commands and environments needed by pandoc snippets
    % extracted from the output of `pandoc -s`
    \providecommand{\tightlist}{%
      \setlength{\itemsep}{0pt}\setlength{\parskip}{0pt}}
    \DefineVerbatimEnvironment{Highlighting}{Verbatim}{commandchars=\\\{\}}
    % Add ',fontsize=\small' for more characters per line
    \newenvironment{Shaded}{}{}
    \newcommand{\KeywordTok}[1]{\textcolor[rgb]{0.00,0.44,0.13}{\textbf{{#1}}}}
    \newcommand{\DataTypeTok}[1]{\textcolor[rgb]{0.56,0.13,0.00}{{#1}}}
    \newcommand{\DecValTok}[1]{\textcolor[rgb]{0.25,0.63,0.44}{{#1}}}
    \newcommand{\BaseNTok}[1]{\textcolor[rgb]{0.25,0.63,0.44}{{#1}}}
    \newcommand{\FloatTok}[1]{\textcolor[rgb]{0.25,0.63,0.44}{{#1}}}
    \newcommand{\CharTok}[1]{\textcolor[rgb]{0.25,0.44,0.63}{{#1}}}
    \newcommand{\StringTok}[1]{\textcolor[rgb]{0.25,0.44,0.63}{{#1}}}
    \newcommand{\CommentTok}[1]{\textcolor[rgb]{0.38,0.63,0.69}{\textit{{#1}}}}
    \newcommand{\OtherTok}[1]{\textcolor[rgb]{0.00,0.44,0.13}{{#1}}}
    \newcommand{\AlertTok}[1]{\textcolor[rgb]{1.00,0.00,0.00}{\textbf{{#1}}}}
    \newcommand{\FunctionTok}[1]{\textcolor[rgb]{0.02,0.16,0.49}{{#1}}}
    \newcommand{\RegionMarkerTok}[1]{{#1}}
    \newcommand{\ErrorTok}[1]{\textcolor[rgb]{1.00,0.00,0.00}{\textbf{{#1}}}}
    \newcommand{\NormalTok}[1]{{#1}}
    
    % Additional commands for more recent versions of Pandoc
    \newcommand{\ConstantTok}[1]{\textcolor[rgb]{0.53,0.00,0.00}{{#1}}}
    \newcommand{\SpecialCharTok}[1]{\textcolor[rgb]{0.25,0.44,0.63}{{#1}}}
    \newcommand{\VerbatimStringTok}[1]{\textcolor[rgb]{0.25,0.44,0.63}{{#1}}}
    \newcommand{\SpecialStringTok}[1]{\textcolor[rgb]{0.73,0.40,0.53}{{#1}}}
    \newcommand{\ImportTok}[1]{{#1}}
    \newcommand{\DocumentationTok}[1]{\textcolor[rgb]{0.73,0.13,0.13}{\textit{{#1}}}}
    \newcommand{\AnnotationTok}[1]{\textcolor[rgb]{0.38,0.63,0.69}{\textbf{\textit{{#1}}}}}
    \newcommand{\CommentVarTok}[1]{\textcolor[rgb]{0.38,0.63,0.69}{\textbf{\textit{{#1}}}}}
    \newcommand{\VariableTok}[1]{\textcolor[rgb]{0.10,0.09,0.49}{{#1}}}
    \newcommand{\ControlFlowTok}[1]{\textcolor[rgb]{0.00,0.44,0.13}{\textbf{{#1}}}}
    \newcommand{\OperatorTok}[1]{\textcolor[rgb]{0.40,0.40,0.40}{{#1}}}
    \newcommand{\BuiltInTok}[1]{{#1}}
    \newcommand{\ExtensionTok}[1]{{#1}}
    \newcommand{\PreprocessorTok}[1]{\textcolor[rgb]{0.74,0.48,0.00}{{#1}}}
    \newcommand{\AttributeTok}[1]{\textcolor[rgb]{0.49,0.56,0.16}{{#1}}}
    \newcommand{\InformationTok}[1]{\textcolor[rgb]{0.38,0.63,0.69}{\textbf{\textit{{#1}}}}}
    \newcommand{\WarningTok}[1]{\textcolor[rgb]{0.38,0.63,0.69}{\textbf{\textit{{#1}}}}}
    
    
    % Define a nice break command that doesn't care if a line doesn't already
    % exist.
    \def\br{\hspace*{\fill} \\* }
    % Math Jax compatibility definitions
    \def\gt{>}
    \def\lt{<}
    \let\Oldtex\TeX
    \let\Oldlatex\LaTeX
    \renewcommand{\TeX}{\textrm{\Oldtex}}
    \renewcommand{\LaTeX}{\textrm{\Oldlatex}}
    % Document parameters
    % Document title
    \title{2.1Prediction1Dregression\_v2}
    
    
    
    
    

    % Pygments definitions
    
\makeatletter
\def\PY@reset{\let\PY@it=\relax \let\PY@bf=\relax%
    \let\PY@ul=\relax \let\PY@tc=\relax%
    \let\PY@bc=\relax \let\PY@ff=\relax}
\def\PY@tok#1{\csname PY@tok@#1\endcsname}
\def\PY@toks#1+{\ifx\relax#1\empty\else%
    \PY@tok{#1}\expandafter\PY@toks\fi}
\def\PY@do#1{\PY@bc{\PY@tc{\PY@ul{%
    \PY@it{\PY@bf{\PY@ff{#1}}}}}}}
\def\PY#1#2{\PY@reset\PY@toks#1+\relax+\PY@do{#2}}

\expandafter\def\csname PY@tok@w\endcsname{\def\PY@tc##1{\textcolor[rgb]{0.73,0.73,0.73}{##1}}}
\expandafter\def\csname PY@tok@c\endcsname{\let\PY@it=\textit\def\PY@tc##1{\textcolor[rgb]{0.25,0.50,0.50}{##1}}}
\expandafter\def\csname PY@tok@cp\endcsname{\def\PY@tc##1{\textcolor[rgb]{0.74,0.48,0.00}{##1}}}
\expandafter\def\csname PY@tok@k\endcsname{\let\PY@bf=\textbf\def\PY@tc##1{\textcolor[rgb]{0.00,0.50,0.00}{##1}}}
\expandafter\def\csname PY@tok@kp\endcsname{\def\PY@tc##1{\textcolor[rgb]{0.00,0.50,0.00}{##1}}}
\expandafter\def\csname PY@tok@kt\endcsname{\def\PY@tc##1{\textcolor[rgb]{0.69,0.00,0.25}{##1}}}
\expandafter\def\csname PY@tok@o\endcsname{\def\PY@tc##1{\textcolor[rgb]{0.40,0.40,0.40}{##1}}}
\expandafter\def\csname PY@tok@ow\endcsname{\let\PY@bf=\textbf\def\PY@tc##1{\textcolor[rgb]{0.67,0.13,1.00}{##1}}}
\expandafter\def\csname PY@tok@nb\endcsname{\def\PY@tc##1{\textcolor[rgb]{0.00,0.50,0.00}{##1}}}
\expandafter\def\csname PY@tok@nf\endcsname{\def\PY@tc##1{\textcolor[rgb]{0.00,0.00,1.00}{##1}}}
\expandafter\def\csname PY@tok@nc\endcsname{\let\PY@bf=\textbf\def\PY@tc##1{\textcolor[rgb]{0.00,0.00,1.00}{##1}}}
\expandafter\def\csname PY@tok@nn\endcsname{\let\PY@bf=\textbf\def\PY@tc##1{\textcolor[rgb]{0.00,0.00,1.00}{##1}}}
\expandafter\def\csname PY@tok@ne\endcsname{\let\PY@bf=\textbf\def\PY@tc##1{\textcolor[rgb]{0.82,0.25,0.23}{##1}}}
\expandafter\def\csname PY@tok@nv\endcsname{\def\PY@tc##1{\textcolor[rgb]{0.10,0.09,0.49}{##1}}}
\expandafter\def\csname PY@tok@no\endcsname{\def\PY@tc##1{\textcolor[rgb]{0.53,0.00,0.00}{##1}}}
\expandafter\def\csname PY@tok@nl\endcsname{\def\PY@tc##1{\textcolor[rgb]{0.63,0.63,0.00}{##1}}}
\expandafter\def\csname PY@tok@ni\endcsname{\let\PY@bf=\textbf\def\PY@tc##1{\textcolor[rgb]{0.60,0.60,0.60}{##1}}}
\expandafter\def\csname PY@tok@na\endcsname{\def\PY@tc##1{\textcolor[rgb]{0.49,0.56,0.16}{##1}}}
\expandafter\def\csname PY@tok@nt\endcsname{\let\PY@bf=\textbf\def\PY@tc##1{\textcolor[rgb]{0.00,0.50,0.00}{##1}}}
\expandafter\def\csname PY@tok@nd\endcsname{\def\PY@tc##1{\textcolor[rgb]{0.67,0.13,1.00}{##1}}}
\expandafter\def\csname PY@tok@s\endcsname{\def\PY@tc##1{\textcolor[rgb]{0.73,0.13,0.13}{##1}}}
\expandafter\def\csname PY@tok@sd\endcsname{\let\PY@it=\textit\def\PY@tc##1{\textcolor[rgb]{0.73,0.13,0.13}{##1}}}
\expandafter\def\csname PY@tok@si\endcsname{\let\PY@bf=\textbf\def\PY@tc##1{\textcolor[rgb]{0.73,0.40,0.53}{##1}}}
\expandafter\def\csname PY@tok@se\endcsname{\let\PY@bf=\textbf\def\PY@tc##1{\textcolor[rgb]{0.73,0.40,0.13}{##1}}}
\expandafter\def\csname PY@tok@sr\endcsname{\def\PY@tc##1{\textcolor[rgb]{0.73,0.40,0.53}{##1}}}
\expandafter\def\csname PY@tok@ss\endcsname{\def\PY@tc##1{\textcolor[rgb]{0.10,0.09,0.49}{##1}}}
\expandafter\def\csname PY@tok@sx\endcsname{\def\PY@tc##1{\textcolor[rgb]{0.00,0.50,0.00}{##1}}}
\expandafter\def\csname PY@tok@m\endcsname{\def\PY@tc##1{\textcolor[rgb]{0.40,0.40,0.40}{##1}}}
\expandafter\def\csname PY@tok@gh\endcsname{\let\PY@bf=\textbf\def\PY@tc##1{\textcolor[rgb]{0.00,0.00,0.50}{##1}}}
\expandafter\def\csname PY@tok@gu\endcsname{\let\PY@bf=\textbf\def\PY@tc##1{\textcolor[rgb]{0.50,0.00,0.50}{##1}}}
\expandafter\def\csname PY@tok@gd\endcsname{\def\PY@tc##1{\textcolor[rgb]{0.63,0.00,0.00}{##1}}}
\expandafter\def\csname PY@tok@gi\endcsname{\def\PY@tc##1{\textcolor[rgb]{0.00,0.63,0.00}{##1}}}
\expandafter\def\csname PY@tok@gr\endcsname{\def\PY@tc##1{\textcolor[rgb]{1.00,0.00,0.00}{##1}}}
\expandafter\def\csname PY@tok@ge\endcsname{\let\PY@it=\textit}
\expandafter\def\csname PY@tok@gs\endcsname{\let\PY@bf=\textbf}
\expandafter\def\csname PY@tok@gp\endcsname{\let\PY@bf=\textbf\def\PY@tc##1{\textcolor[rgb]{0.00,0.00,0.50}{##1}}}
\expandafter\def\csname PY@tok@go\endcsname{\def\PY@tc##1{\textcolor[rgb]{0.53,0.53,0.53}{##1}}}
\expandafter\def\csname PY@tok@gt\endcsname{\def\PY@tc##1{\textcolor[rgb]{0.00,0.27,0.87}{##1}}}
\expandafter\def\csname PY@tok@err\endcsname{\def\PY@bc##1{\setlength{\fboxsep}{0pt}\fcolorbox[rgb]{1.00,0.00,0.00}{1,1,1}{\strut ##1}}}
\expandafter\def\csname PY@tok@kc\endcsname{\let\PY@bf=\textbf\def\PY@tc##1{\textcolor[rgb]{0.00,0.50,0.00}{##1}}}
\expandafter\def\csname PY@tok@kd\endcsname{\let\PY@bf=\textbf\def\PY@tc##1{\textcolor[rgb]{0.00,0.50,0.00}{##1}}}
\expandafter\def\csname PY@tok@kn\endcsname{\let\PY@bf=\textbf\def\PY@tc##1{\textcolor[rgb]{0.00,0.50,0.00}{##1}}}
\expandafter\def\csname PY@tok@kr\endcsname{\let\PY@bf=\textbf\def\PY@tc##1{\textcolor[rgb]{0.00,0.50,0.00}{##1}}}
\expandafter\def\csname PY@tok@bp\endcsname{\def\PY@tc##1{\textcolor[rgb]{0.00,0.50,0.00}{##1}}}
\expandafter\def\csname PY@tok@fm\endcsname{\def\PY@tc##1{\textcolor[rgb]{0.00,0.00,1.00}{##1}}}
\expandafter\def\csname PY@tok@vc\endcsname{\def\PY@tc##1{\textcolor[rgb]{0.10,0.09,0.49}{##1}}}
\expandafter\def\csname PY@tok@vg\endcsname{\def\PY@tc##1{\textcolor[rgb]{0.10,0.09,0.49}{##1}}}
\expandafter\def\csname PY@tok@vi\endcsname{\def\PY@tc##1{\textcolor[rgb]{0.10,0.09,0.49}{##1}}}
\expandafter\def\csname PY@tok@vm\endcsname{\def\PY@tc##1{\textcolor[rgb]{0.10,0.09,0.49}{##1}}}
\expandafter\def\csname PY@tok@sa\endcsname{\def\PY@tc##1{\textcolor[rgb]{0.73,0.13,0.13}{##1}}}
\expandafter\def\csname PY@tok@sb\endcsname{\def\PY@tc##1{\textcolor[rgb]{0.73,0.13,0.13}{##1}}}
\expandafter\def\csname PY@tok@sc\endcsname{\def\PY@tc##1{\textcolor[rgb]{0.73,0.13,0.13}{##1}}}
\expandafter\def\csname PY@tok@dl\endcsname{\def\PY@tc##1{\textcolor[rgb]{0.73,0.13,0.13}{##1}}}
\expandafter\def\csname PY@tok@s2\endcsname{\def\PY@tc##1{\textcolor[rgb]{0.73,0.13,0.13}{##1}}}
\expandafter\def\csname PY@tok@sh\endcsname{\def\PY@tc##1{\textcolor[rgb]{0.73,0.13,0.13}{##1}}}
\expandafter\def\csname PY@tok@s1\endcsname{\def\PY@tc##1{\textcolor[rgb]{0.73,0.13,0.13}{##1}}}
\expandafter\def\csname PY@tok@mb\endcsname{\def\PY@tc##1{\textcolor[rgb]{0.40,0.40,0.40}{##1}}}
\expandafter\def\csname PY@tok@mf\endcsname{\def\PY@tc##1{\textcolor[rgb]{0.40,0.40,0.40}{##1}}}
\expandafter\def\csname PY@tok@mh\endcsname{\def\PY@tc##1{\textcolor[rgb]{0.40,0.40,0.40}{##1}}}
\expandafter\def\csname PY@tok@mi\endcsname{\def\PY@tc##1{\textcolor[rgb]{0.40,0.40,0.40}{##1}}}
\expandafter\def\csname PY@tok@il\endcsname{\def\PY@tc##1{\textcolor[rgb]{0.40,0.40,0.40}{##1}}}
\expandafter\def\csname PY@tok@mo\endcsname{\def\PY@tc##1{\textcolor[rgb]{0.40,0.40,0.40}{##1}}}
\expandafter\def\csname PY@tok@ch\endcsname{\let\PY@it=\textit\def\PY@tc##1{\textcolor[rgb]{0.25,0.50,0.50}{##1}}}
\expandafter\def\csname PY@tok@cm\endcsname{\let\PY@it=\textit\def\PY@tc##1{\textcolor[rgb]{0.25,0.50,0.50}{##1}}}
\expandafter\def\csname PY@tok@cpf\endcsname{\let\PY@it=\textit\def\PY@tc##1{\textcolor[rgb]{0.25,0.50,0.50}{##1}}}
\expandafter\def\csname PY@tok@c1\endcsname{\let\PY@it=\textit\def\PY@tc##1{\textcolor[rgb]{0.25,0.50,0.50}{##1}}}
\expandafter\def\csname PY@tok@cs\endcsname{\let\PY@it=\textit\def\PY@tc##1{\textcolor[rgb]{0.25,0.50,0.50}{##1}}}

\def\PYZbs{\char`\\}
\def\PYZus{\char`\_}
\def\PYZob{\char`\{}
\def\PYZcb{\char`\}}
\def\PYZca{\char`\^}
\def\PYZam{\char`\&}
\def\PYZlt{\char`\<}
\def\PYZgt{\char`\>}
\def\PYZsh{\char`\#}
\def\PYZpc{\char`\%}
\def\PYZdl{\char`\$}
\def\PYZhy{\char`\-}
\def\PYZsq{\char`\'}
\def\PYZdq{\char`\"}
\def\PYZti{\char`\~}
% for compatibility with earlier versions
\def\PYZat{@}
\def\PYZlb{[}
\def\PYZrb{]}
\makeatother


    % Exact colors from NB
    \definecolor{incolor}{rgb}{0.0, 0.0, 0.5}
    \definecolor{outcolor}{rgb}{0.545, 0.0, 0.0}



    
    % Prevent overflowing lines due to hard-to-break entities
    \sloppy 
    % Setup hyperref package
    \hypersetup{
      breaklinks=true,  % so long urls are correctly broken across lines
      colorlinks=true,
      urlcolor=urlcolor,
      linkcolor=linkcolor,
      citecolor=citecolor,
      }
    % Slightly bigger margins than the latex defaults
    
    \geometry{verbose,tmargin=1in,bmargin=1in,lmargin=1in,rmargin=1in}
    
    

    \begin{document}
    
    
    \maketitle
    
    

    
     

    

    Linear Regression 1D: Prediction

    Table of Contents

In this lab, we will review how to make a prediction in several
different ways by using PyTorch.

\begin{verbatim}
<li><a href="#Prediction">Prediction</a></li>
<li><a href="#Linear">Class Linear</a></li>
<li><a href="#Cust">Build Custom Modules</a></li>
\end{verbatim}

Estimated Time Needed: 15 min

    Preparation

    The following are the libraries we are going to use for this lab.

    \begin{Verbatim}[commandchars=\\\{\}]
{\color{incolor}In [{\color{incolor}1}]:} \PY{c+c1}{\PYZsh{} These are the libraries will be used for this lab.}
        
        \PY{k+kn}{import} \PY{n+nn}{torch}
\end{Verbatim}

    

    Prediction

    Let us create the following expressions:

    \(b=-1,w=2\)

\(\hat{y}=-1+2x\)

    First, define the parameters:

    \begin{Verbatim}[commandchars=\\\{\}]
{\color{incolor}In [{\color{incolor}2}]:} \PY{c+c1}{\PYZsh{} Define w = 2 and b = \PYZhy{}1 for y = wx + b}
        
        \PY{n}{w} \PY{o}{=} \PY{n}{torch}\PY{o}{.}\PY{n}{tensor}\PY{p}{(}\PY{l+m+mf}{2.0}\PY{p}{,} \PY{n}{requires\PYZus{}grad} \PY{o}{=} \PY{k+kc}{True}\PY{p}{)}
        \PY{n}{b} \PY{o}{=} \PY{n}{torch}\PY{o}{.}\PY{n}{tensor}\PY{p}{(}\PY{o}{\PYZhy{}}\PY{l+m+mf}{1.0}\PY{p}{,} \PY{n}{requires\PYZus{}grad} \PY{o}{=} \PY{k+kc}{True}\PY{p}{)}
\end{Verbatim}

    Then, define the function forward(x, w, b) makes the prediction:

    \begin{Verbatim}[commandchars=\\\{\}]
{\color{incolor}In [{\color{incolor}3}]:} \PY{c+c1}{\PYZsh{} Function forward(x) for prediction}
        
        \PY{k}{def} \PY{n+nf}{forward}\PY{p}{(}\PY{n}{x}\PY{p}{)}\PY{p}{:}
            \PY{n}{yhat} \PY{o}{=} \PY{n}{w} \PY{o}{*} \PY{n}{x} \PY{o}{+} \PY{n}{b}
            \PY{k}{return} \PY{n}{yhat}
\end{Verbatim}

    Let's make the following prediction at x = 1

    \(\hat{y}=-1+2x\)

\(\hat{y}=-1+2(1)\)

    \begin{Verbatim}[commandchars=\\\{\}]
{\color{incolor}In [{\color{incolor}4}]:} \PY{c+c1}{\PYZsh{} Predict y = 2x \PYZhy{} 1 at x = 1}
        
        \PY{n}{x} \PY{o}{=} \PY{n}{torch}\PY{o}{.}\PY{n}{tensor}\PY{p}{(}\PY{p}{[}\PY{p}{[}\PY{l+m+mf}{1.0}\PY{p}{]}\PY{p}{]}\PY{p}{)}
        \PY{n}{yhat} \PY{o}{=} \PY{n}{forward}\PY{p}{(}\PY{n}{x}\PY{p}{)}
        \PY{n+nb}{print}\PY{p}{(}\PY{l+s+s2}{\PYZdq{}}\PY{l+s+s2}{The prediction: }\PY{l+s+s2}{\PYZdq{}}\PY{p}{,} \PY{n}{yhat}\PY{p}{)}
\end{Verbatim}

    \begin{Verbatim}[commandchars=\\\{\}]
The prediction:  tensor([[1.]], grad\_fn=<ThAddBackward>)

    \end{Verbatim}

    

    Now, let us try to make the prediction for multiple inputs:

    

    Let us construct the x tensor first. Check the shape of x.

    \begin{Verbatim}[commandchars=\\\{\}]
{\color{incolor}In [{\color{incolor}5}]:} \PY{c+c1}{\PYZsh{} Create x Tensor and check the shape of x tensor}
        
        \PY{n}{x} \PY{o}{=} \PY{n}{torch}\PY{o}{.}\PY{n}{tensor}\PY{p}{(}\PY{p}{[}\PY{p}{[}\PY{l+m+mf}{1.0}\PY{p}{]}\PY{p}{,} \PY{p}{[}\PY{l+m+mf}{2.0}\PY{p}{]}\PY{p}{]}\PY{p}{)}
        \PY{n+nb}{print}\PY{p}{(}\PY{l+s+s2}{\PYZdq{}}\PY{l+s+s2}{The shape of x: }\PY{l+s+s2}{\PYZdq{}}\PY{p}{,} \PY{n}{x}\PY{o}{.}\PY{n}{shape}\PY{p}{)}
\end{Verbatim}

    \begin{Verbatim}[commandchars=\\\{\}]
The shape of x:  torch.Size([2, 1])

    \end{Verbatim}

    Now make the prediction:

    \begin{Verbatim}[commandchars=\\\{\}]
{\color{incolor}In [{\color{incolor}6}]:} \PY{c+c1}{\PYZsh{} Make the prediction of y = 2x \PYZhy{} 1 at x = [1, 2]}
        
        \PY{n}{yhat} \PY{o}{=} \PY{n}{forward}\PY{p}{(}\PY{n}{x}\PY{p}{)}
        \PY{n+nb}{print}\PY{p}{(}\PY{l+s+s2}{\PYZdq{}}\PY{l+s+s2}{The prediction: }\PY{l+s+s2}{\PYZdq{}}\PY{p}{,} \PY{n}{yhat}\PY{p}{)}
\end{Verbatim}

    \begin{Verbatim}[commandchars=\\\{\}]
The prediction:  tensor([[1.],
        [3.]], grad\_fn=<ThAddBackward>)

    \end{Verbatim}

    The result is the same as what it is in the image above.

    

    Practice

    Make a prediction of the following x tensor using the w and b from
above.

    \begin{Verbatim}[commandchars=\\\{\}]
{\color{incolor}In [{\color{incolor}7}]:} \PY{c+c1}{\PYZsh{} Practice: Make a prediction of y = 2x \PYZhy{} 1 at x = [[1.0], [2.0], [3.0]]}
        
        \PY{n}{x} \PY{o}{=} \PY{n}{torch}\PY{o}{.}\PY{n}{tensor}\PY{p}{(}\PY{p}{[}\PY{p}{[}\PY{l+m+mf}{1.0}\PY{p}{]}\PY{p}{,} \PY{p}{[}\PY{l+m+mf}{2.0}\PY{p}{]}\PY{p}{,} \PY{p}{[}\PY{l+m+mf}{3.0}\PY{p}{]}\PY{p}{]}\PY{p}{)}
        \PY{n}{yhat} \PY{o}{=} \PY{n}{forward}\PY{p}{(}\PY{n}{x}\PY{p}{)}
        \PY{n+nb}{print}\PY{p}{(}\PY{n}{yhat}\PY{p}{)}
\end{Verbatim}

    \begin{Verbatim}[commandchars=\\\{\}]
tensor([[1.],
        [3.],
        [5.]], grad\_fn=<ThAddBackward>)

    \end{Verbatim}

    Double-click here for the solution.

    

    Class Linear

    The linear class can be used to make a prediction. We can also use the
linear class to build more complex models. Let's import the module:

    \begin{Verbatim}[commandchars=\\\{\}]
{\color{incolor}In [{\color{incolor}8}]:} \PY{c+c1}{\PYZsh{} Import Class Linear}
        
        \PY{k+kn}{from} \PY{n+nn}{torch}\PY{n+nn}{.}\PY{n+nn}{nn} \PY{k}{import} \PY{n}{Linear}
\end{Verbatim}

    Set the random seed because the parameters are randomly initialized:

    \begin{Verbatim}[commandchars=\\\{\}]
{\color{incolor}In [{\color{incolor}9}]:} \PY{c+c1}{\PYZsh{} Set random seed}
        
        \PY{n}{torch}\PY{o}{.}\PY{n}{manual\PYZus{}seed}\PY{p}{(}\PY{l+m+mi}{1}\PY{p}{)}
\end{Verbatim}

\begin{Verbatim}[commandchars=\\\{\}]
{\color{outcolor}Out[{\color{outcolor}9}]:} <torch.\_C.Generator at 0x7f00033575d0>
\end{Verbatim}
            
    

    Let us create the linear object by using the constructor. The parameters
are randomly created. Let us print out to see what w and b are.

    \begin{Verbatim}[commandchars=\\\{\}]
{\color{incolor}In [{\color{incolor}10}]:} \PY{c+c1}{\PYZsh{} Create Linear Regression Model, and print out the parameters}
         
         \PY{n}{lr} \PY{o}{=} \PY{n}{Linear}\PY{p}{(}\PY{n}{in\PYZus{}features}\PY{o}{=}\PY{l+m+mi}{1}\PY{p}{,} \PY{n}{out\PYZus{}features}\PY{o}{=}\PY{l+m+mi}{1}\PY{p}{,} \PY{n}{bias}\PY{o}{=}\PY{k+kc}{True}\PY{p}{)}
         \PY{n+nb}{print}\PY{p}{(}\PY{l+s+s2}{\PYZdq{}}\PY{l+s+s2}{Parameters w and b: }\PY{l+s+s2}{\PYZdq{}}\PY{p}{,} \PY{n+nb}{list}\PY{p}{(}\PY{n}{lr}\PY{o}{.}\PY{n}{parameters}\PY{p}{(}\PY{p}{)}\PY{p}{)}\PY{p}{)}
\end{Verbatim}

    \begin{Verbatim}[commandchars=\\\{\}]
Parameters w and b:  [Parameter containing:
tensor([[0.5153]], requires\_grad=True), Parameter containing:
tensor([-0.4414], requires\_grad=True)]

    \end{Verbatim}

    This is equivalent to the following expression:

    \(b=-0.44, w=0.5153\)

\(\hat{y}=-0.44+0.5153x\)

    Now let us make a single prediction at x = {[}{[}1.0{]}{]}.

    \begin{Verbatim}[commandchars=\\\{\}]
{\color{incolor}In [{\color{incolor}11}]:} \PY{c+c1}{\PYZsh{} Make the prediction at x = [[1.0]]}
         
         \PY{n}{x} \PY{o}{=} \PY{n}{torch}\PY{o}{.}\PY{n}{tensor}\PY{p}{(}\PY{p}{[}\PY{p}{[}\PY{l+m+mf}{1.0}\PY{p}{]}\PY{p}{]}\PY{p}{)}
         \PY{n}{yhat} \PY{o}{=} \PY{n}{lr}\PY{p}{(}\PY{n}{x}\PY{p}{)}
         \PY{n+nb}{print}\PY{p}{(}\PY{l+s+s2}{\PYZdq{}}\PY{l+s+s2}{The prediction: }\PY{l+s+s2}{\PYZdq{}}\PY{p}{,} \PY{n}{yhat}\PY{p}{)}
\end{Verbatim}

    \begin{Verbatim}[commandchars=\\\{\}]
The prediction:  tensor([[0.0739]], grad\_fn=<ThAddmmBackward>)

    \end{Verbatim}

    

    Similarly, you can make multiple predictions:

    

    Use model lr(x) to predict the result.

    \begin{Verbatim}[commandchars=\\\{\}]
{\color{incolor}In [{\color{incolor}12}]:} \PY{c+c1}{\PYZsh{} Create the prediction using linear model}
         
         \PY{n}{x} \PY{o}{=} \PY{n}{torch}\PY{o}{.}\PY{n}{tensor}\PY{p}{(}\PY{p}{[}\PY{p}{[}\PY{l+m+mf}{1.0}\PY{p}{]}\PY{p}{,} \PY{p}{[}\PY{l+m+mf}{2.0}\PY{p}{]}\PY{p}{]}\PY{p}{)}
         \PY{n}{yhat} \PY{o}{=} \PY{n}{lr}\PY{p}{(}\PY{n}{x}\PY{p}{)}
         \PY{n+nb}{print}\PY{p}{(}\PY{l+s+s2}{\PYZdq{}}\PY{l+s+s2}{The prediction: }\PY{l+s+s2}{\PYZdq{}}\PY{p}{,} \PY{n}{yhat}\PY{p}{)}
\end{Verbatim}

    \begin{Verbatim}[commandchars=\\\{\}]
The prediction:  tensor([[0.0739],
        [0.5891]], grad\_fn=<ThAddmmBackward>)

    \end{Verbatim}

    

    Practice

    Make a prediction of the following x tensor using the linear regression
model lr.

    \begin{Verbatim}[commandchars=\\\{\}]
{\color{incolor}In [{\color{incolor}13}]:} \PY{c+c1}{\PYZsh{} Practice: Use the linear regression model object lr to make the prediction.}
         
         \PY{n}{x} \PY{o}{=} \PY{n}{torch}\PY{o}{.}\PY{n}{tensor}\PY{p}{(}\PY{p}{[}\PY{p}{[}\PY{l+m+mf}{1.0}\PY{p}{]}\PY{p}{,}\PY{p}{[}\PY{l+m+mf}{2.0}\PY{p}{]}\PY{p}{,}\PY{p}{[}\PY{l+m+mf}{3.0}\PY{p}{]}\PY{p}{]}\PY{p}{)}
         \PY{n}{yhat} \PY{o}{=} \PY{n}{lr}\PY{p}{(}\PY{n}{x}\PY{p}{)}
         \PY{n+nb}{print}\PY{p}{(}\PY{l+s+s2}{\PYZdq{}}\PY{l+s+s2}{The prediction: }\PY{l+s+s2}{\PYZdq{}}\PY{p}{,} \PY{n}{yhat}\PY{p}{)}
\end{Verbatim}

    \begin{Verbatim}[commandchars=\\\{\}]
The prediction:  tensor([[0.0739],
        [0.5891],
        [1.1044]], grad\_fn=<ThAddmmBackward>)

    \end{Verbatim}

    Double-click here for the solution.

    

    Build Custom Modules

    Now, let's build a custom module. We can make more complex models by
using this method later on.

    First, import the following library.

    \begin{Verbatim}[commandchars=\\\{\}]
{\color{incolor}In [{\color{incolor}14}]:} \PY{c+c1}{\PYZsh{} Library for this section}
         
         \PY{k+kn}{from} \PY{n+nn}{torch} \PY{k}{import} \PY{n}{nn}
\end{Verbatim}

    Now, let us define the class:

    \begin{Verbatim}[commandchars=\\\{\}]
{\color{incolor}In [{\color{incolor}15}]:} \PY{c+c1}{\PYZsh{} Customize Linear Regression Class}
         
         \PY{k}{class} \PY{n+nc}{LR}\PY{p}{(}\PY{n}{nn}\PY{o}{.}\PY{n}{Module}\PY{p}{)}\PY{p}{:}
             
             \PY{c+c1}{\PYZsh{} Constructor}
             \PY{k}{def} \PY{n+nf}{\PYZus{}\PYZus{}init\PYZus{}\PYZus{}}\PY{p}{(}\PY{n+nb+bp}{self}\PY{p}{,} \PY{n}{input\PYZus{}size}\PY{p}{,} \PY{n}{output\PYZus{}size}\PY{p}{)}\PY{p}{:}
                 
                 \PY{c+c1}{\PYZsh{} Inherit from parent}
                 \PY{n+nb}{super}\PY{p}{(}\PY{n}{LR}\PY{p}{,} \PY{n+nb+bp}{self}\PY{p}{)}\PY{o}{.}\PY{n+nf+fm}{\PYZus{}\PYZus{}init\PYZus{}\PYZus{}}\PY{p}{(}\PY{p}{)}
                 \PY{n+nb+bp}{self}\PY{o}{.}\PY{n}{linear} \PY{o}{=} \PY{n}{nn}\PY{o}{.}\PY{n}{Linear}\PY{p}{(}\PY{n}{input\PYZus{}size}\PY{p}{,} \PY{n}{output\PYZus{}size}\PY{p}{)}
             
             \PY{c+c1}{\PYZsh{} Prediction function}
             \PY{k}{def} \PY{n+nf}{forward}\PY{p}{(}\PY{n+nb+bp}{self}\PY{p}{,} \PY{n}{x}\PY{p}{)}\PY{p}{:}
                 \PY{n}{out} \PY{o}{=} \PY{n+nb+bp}{self}\PY{o}{.}\PY{n}{linear}\PY{p}{(}\PY{n}{x}\PY{p}{)}
                 \PY{k}{return} \PY{n}{out}
\end{Verbatim}

    Create an object by using the constructor. Print out the parameters we
get and the model.

    \begin{Verbatim}[commandchars=\\\{\}]
{\color{incolor}In [{\color{incolor}16}]:} \PY{c+c1}{\PYZsh{} Create the linear regression model. Print out the parameters.}
         
         \PY{n}{lr} \PY{o}{=} \PY{n}{LR}\PY{p}{(}\PY{l+m+mi}{1}\PY{p}{,} \PY{l+m+mi}{1}\PY{p}{)}
         \PY{n+nb}{print}\PY{p}{(}\PY{l+s+s2}{\PYZdq{}}\PY{l+s+s2}{The parameters: }\PY{l+s+s2}{\PYZdq{}}\PY{p}{,} \PY{n+nb}{list}\PY{p}{(}\PY{n}{lr}\PY{o}{.}\PY{n}{parameters}\PY{p}{(}\PY{p}{)}\PY{p}{)}\PY{p}{)}
         \PY{n+nb}{print}\PY{p}{(}\PY{l+s+s2}{\PYZdq{}}\PY{l+s+s2}{Linear model: }\PY{l+s+s2}{\PYZdq{}}\PY{p}{,} \PY{n}{lr}\PY{o}{.}\PY{n}{linear}\PY{p}{)}
\end{Verbatim}

    \begin{Verbatim}[commandchars=\\\{\}]
The parameters:  [Parameter containing:
tensor([[-0.1939]], requires\_grad=True), Parameter containing:
tensor([0.4694], requires\_grad=True)]
Linear model:  Linear(in\_features=1, out\_features=1, bias=True)

    \end{Verbatim}

    

    Let us try to make a prediction of a single input sample.

    \begin{Verbatim}[commandchars=\\\{\}]
{\color{incolor}In [{\color{incolor}17}]:} \PY{c+c1}{\PYZsh{} Try our customize linear regression model with single input}
         
         \PY{n}{x} \PY{o}{=} \PY{n}{torch}\PY{o}{.}\PY{n}{tensor}\PY{p}{(}\PY{p}{[}\PY{p}{[}\PY{l+m+mf}{1.0}\PY{p}{]}\PY{p}{]}\PY{p}{)}
         \PY{n}{yhat} \PY{o}{=} \PY{n}{lr}\PY{p}{(}\PY{n}{x}\PY{p}{)}
         \PY{n+nb}{print}\PY{p}{(}\PY{l+s+s2}{\PYZdq{}}\PY{l+s+s2}{The prediction: }\PY{l+s+s2}{\PYZdq{}}\PY{p}{,} \PY{n}{yhat}\PY{p}{)}
\end{Verbatim}

    \begin{Verbatim}[commandchars=\\\{\}]
The prediction:  tensor([[0.2755]], grad\_fn=<ThAddmmBackward>)

    \end{Verbatim}

    

    Now, let us try another example with multiple samples.

    \begin{Verbatim}[commandchars=\\\{\}]
{\color{incolor}In [{\color{incolor}18}]:} \PY{c+c1}{\PYZsh{} Try our customize linear regression model with multiple input}
         
         \PY{n}{x} \PY{o}{=} \PY{n}{torch}\PY{o}{.}\PY{n}{tensor}\PY{p}{(}\PY{p}{[}\PY{p}{[}\PY{l+m+mf}{1.0}\PY{p}{]}\PY{p}{,} \PY{p}{[}\PY{l+m+mf}{2.0}\PY{p}{]}\PY{p}{]}\PY{p}{)}
         \PY{n}{yhat} \PY{o}{=} \PY{n}{lr}\PY{p}{(}\PY{n}{x}\PY{p}{)}
         \PY{n+nb}{print}\PY{p}{(}\PY{l+s+s2}{\PYZdq{}}\PY{l+s+s2}{The prediction: }\PY{l+s+s2}{\PYZdq{}}\PY{p}{,} \PY{n}{yhat}\PY{p}{)}
\end{Verbatim}

    \begin{Verbatim}[commandchars=\\\{\}]
The prediction:  tensor([[0.2755],
        [0.0816]], grad\_fn=<ThAddmmBackward>)

    \end{Verbatim}

    

    Practice

    Create an object lr1 from the class we created before and make a
prediction by using the following tensor:

    \begin{Verbatim}[commandchars=\\\{\}]
{\color{incolor}In [{\color{incolor}19}]:} \PY{c+c1}{\PYZsh{} Practice: Use the LR class to create a model and make a prediction of the following tensor.}
         
         \PY{n}{x} \PY{o}{=} \PY{n}{torch}\PY{o}{.}\PY{n}{tensor}\PY{p}{(}\PY{p}{[}\PY{p}{[}\PY{l+m+mf}{1.0}\PY{p}{]}\PY{p}{,} \PY{p}{[}\PY{l+m+mf}{2.0}\PY{p}{]}\PY{p}{,} \PY{p}{[}\PY{l+m+mf}{3.0}\PY{p}{]}\PY{p}{]}\PY{p}{)}
         \PY{n}{yhat} \PY{o}{=} \PY{n}{lr}\PY{p}{(}\PY{n}{x}\PY{p}{)}
         \PY{n+nb}{print}\PY{p}{(}\PY{l+s+s2}{\PYZdq{}}\PY{l+s+s2}{The prediction: }\PY{l+s+s2}{\PYZdq{}}\PY{p}{,} \PY{n}{yhat}\PY{p}{)}
\end{Verbatim}

    \begin{Verbatim}[commandchars=\\\{\}]
The prediction:  tensor([[ 0.2755],
        [ 0.0816],
        [-0.1122]], grad\_fn=<ThAddmmBackward>)

    \end{Verbatim}

    Double-click here for the solution.

    

     

    About the Authors:

Joseph Santarcangelo has a PhD in Electrical Engineering, his research
focused on using machine learning, signal processing, and computer
vision to determine how videos impact human cognition. Joseph has been
working for IBM since he completed his PhD.

    Other contributors: Michelle Carey, Mavis Zhou

    

    Copyright © 2018 cognitiveclass.ai. This notebook and its source code
are released under the terms of the MIT License.


    % Add a bibliography block to the postdoc
    
    
    
    \end{document}
